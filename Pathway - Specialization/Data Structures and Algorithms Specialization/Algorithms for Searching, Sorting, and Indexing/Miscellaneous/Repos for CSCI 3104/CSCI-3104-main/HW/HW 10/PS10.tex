 \documentclass[11pt]{article} 
\usepackage[english]{babel}
\usepackage[utf8]{inputenc}
\usepackage[margin=0.5in]{geometry}
\usepackage{amsmath}
\usepackage{amsthm}
\usepackage{amsfonts}
\usepackage{amssymb}
\usepackage[usenames,dvipsnames]{xcolor}
\usepackage{graphicx}
\usepackage[siunitx]{circuitikz}
\usepackage{tikz}
\usepackage{tkz-berge}
\usetikzlibrary{positioning, automata, backgrounds}
\usepackage[colorinlistoftodos, color=orange!50]{todonotes}
\usepackage{hyperref}
\usepackage[numbers, square]{natbib}
\usepackage{fancybox}
\usepackage{epsfig}
\usepackage{soul}
\usepackage[framemethod=tikz]{mdframed}
\usepackage[shortlabels]{enumitem}
\usepackage[version=4]{mhchem}
\usepackage{multicol}
\usepackage{forest}
\usepackage{mathtools}
\usepackage{comment}
\usepackage{enumitem}
\usepackage[utf8]{inputenc}
\usepackage{listings}
\usepackage{color}
\usepackage[numbers]{natbib}
\usepackage{subfiles}
\usepackage{tkz-berge}
\usepackage{algorithm}
\usepackage[noend]{algpseudocode}


\newtheorem{prop}{Proposition}[section]
\newtheorem{thm}{Theorem}[section]
\newtheorem{lemma}{Lemma}[section]
\newtheorem{cor}{Corollary}[prop]

\theoremstyle{definition}
\newtheorem{definition}{Definition}

\theoremstyle{definition}
\newtheorem{required}{Problem}

\theoremstyle{definition}
\newtheorem{ex}{Example}

\newcommand{\interval}[4]{\draw (#2, #1) -- (#3, #1); % Usage: \interval{height}{start}{end}{label}
\draw (#2, #1-0.11) -- (#2, #1+0.11); % draw left whisker
\draw (#3, #1-0.11) -- (#3, #1+0.11); % draw right whisker
\node[] at (#2-0.25, #1) {#4};
}


\setlength{\marginparwidth}{3.4cm}
%#########################################################

%To use symbols for footnotes
\renewcommand*{\thefootnote}{\fnsymbol{footnote}}
%To change footnotes back to numbers uncomment the following line
%\renewcommand*{\thefootnote}{\arabic{footnote}}

% Enable this command to adjust line spacing for inline math equations.
% \everymath{\displaystyle}

% _______ _____ _______ _      ______ 
%|__   __|_   _|__   __| |    |  ____|
%   | |    | |    | |  | |    | |__   
%   | |    | |    | |  | |    |  __|  
%   | |   _| |_   | |  | |____| |____ 
%   |_|  |_____|  |_|  |______|______|
%%%%%%%%%%%%%%%%%%%%%%%%%%%%%%%%%%%%%%%

\title{
\normalfont \normalsize 
\textsc{CSCI 3104 Fall 2021 \\ 
Instructors: Profs. Chen and Layer} \\
[10pt] 
\rule{\linewidth}{0.5pt} \\[6pt] 
\huge Problem Set 10 \\
\rule{\linewidth}{2pt}  \\[10pt]
}
%\author{Your Name}
\date{}

\begin{document}
\definecolor {processblue}{cmyk}{0.96,0,0,0}
\maketitle


%%%%%%%%%%%%%%%%%%%%%%%%%
%%%%%%%%%%%%%%%%%%%%%%%%%%
%%%%%%%%%%FILL IN YOUR NAME%%%%%%%
%%%%%%%%%%AND STUDENT ID%%%%%%%%
%%%%%%%%%%%%%%%%%%%%%%%%%%
\noindent
Due Date \dotfill April 26 \\
Name \dotfill \textbf{Your Name} \\
Student ID \dotfill \textbf{Your Student ID} \\
Collaborators \dotfill \textbf{List Your Collaborators Here}

\tableofcontents

\section{Instructions}
 \begin{itemize}
	\item The solutions \textbf{must be typed}, using proper mathematical notation. We cannot accept hand-written solutions. \href{http://ece.uprm.edu/~caceros/latex/introduction.pdf}{Here's a short intro to \LaTeX.}
	\item You should submit your work through the \textbf{class Canvas page} only. Please submit one PDF file, compiled using this \LaTeX \ template.
	\item You may not need a full page for your solutions; pagebreaks are there to help Gradescope automatically find where each problem is. Even if you do not attempt every problem, please submit this document with no fewer pages than the blank template (or Gradescope has issues with it).

	\item You are welcome and encouraged to collaborate with your classmates, as well as consult outside resources. You must \textbf{cite your sources in this document.} \textbf{Copying from any source is an Honor Code violation. Furthermore, all submissions must be in your own words and reflect your understanding of the material.} If there is any confusion about this policy, it is your responsibility to clarify before the due date. 

	\item Posting to \textbf{any} service including, but not limited to Chegg, Reddit, StackExchange, etc., for help on an assignment is a violation of the Honor Code.
\end{itemize}


\newpage
\section{Standard 26 - Showing problems belong to $\textsf{P}$}
\begin{required} \label{InP}
Consider the Shortest Path problem that takes as input a graph $G=(V,E)$
and two vertices $v,t\in V$ and returns the shortest path from $v$ to $t$.
The shortest path decision problem takes as input a graph $G=(V,E)$, two a
vertices $v,t\in V$, and a value $k$, and returns True if there is a path from
$v$ to $t$ that is at most $k$ edges and False otherwise. Show that the
shortest path decision problem is in $\textsf{P}$. You are welcome and
encouraged to cite algorithms we have previously covered in class, including
known facts about their runtime.  [\textbf{Note:} To gauge the level of
detail, we expect your solutions to this problem will be 2-4 sentences. We are
not asking you to analyze an algorithm in great detail.]

\end{required}
\begin{proof}[Answer]

\end{proof}


\newpage
\section{Standard 27 - Showing problems belong to $\textsf{NP}$}
\begin{required} \label{InNP}
Consider the Simple Shortest Path decision problem that takes as input a
directed graph $G=(V,E)$, a cost function $c(e)\in \mathbb{Z}$ for $e \in E$,
and two vertices $v,t\in V$. The problem returns True if there is a simple path from
$v$ to $t$ with edge weights that sum to at most $k$, and False otherwise.
Show this problem is in $\textsf{NP}$.
\end{required}

\begin{proof}[Answer]

\end{proof}

\newpage
\begin{required} 
Indiana Jones is gathering $n$ artifacts from a tomb, which is about to crumble and needs to fit them into $5$ cases. Each case can carry up to $W$ kilograms, where $W$ is fixed. Suppose the weight of artifact $i$ is the positive integer $w_{i}$. Indiana Jones needs to decide if he is able to pack all the artifacts. We formalize the \textsf{Indiana Jones} decision problem as follows.
\begin{itemize}
\item \textsf{Instance:} The weights of our $n$ items, $w_{1}, \ldots, w_{n} > 0$. 
\item \textsf{Decision:} Is there a way to place the $n$ items into different cases, such that each case is carrying weight at most $W$?
\end{itemize}

\noindent Show that $\textsf{Indiana Jones} \in \textsf{NP}.$
\end{required}

\begin{proof}[Answer]

\end{proof}

\newpage
\section{Standard 27 - $\textsf{NP}$-compelteness: Reduction}
\begin{required} \label{S30Prob1}
A student has a decision problem $L$ which they know is in the class \textsf{NP}. This student wishes to show that L is \textsf{NP}-complete. They attempt to do so by constructing a polynomial time reduction from $L$ to \textsf{SAT}, a known \textsf{NP}-complete problem. That is, the student attempts to show that $L \leq_{p} \textsf{SAT}.$ Determine if this student’s approach is
correct and justify your answer.
\end{required}

\begin{proof}[Answer]

\end{proof}



\newpage
\begin{required} \label{NPcomp}
Consider the Simple Shortest Path decision problem that takes as input a
directed graph $G=(V,E)$, a cost function $c(e)\in \mathbb{Z}$ for $e \in E$,
and two vertices $v,t\in V$. The problem returns True if there is a simple path from
$v$ to $t$ with edge weights that sum to at most $k$, and False otherwise.
Show this problem is $\textsf{NP}$-compelte.
\end{required}

\begin{proof}[Answer]

\end{proof}


\end{document} % NOTHING AFTER THIS LINE IS PART OF THE DOCUMENT



