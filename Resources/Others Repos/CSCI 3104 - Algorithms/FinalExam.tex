\documentclass[12pt]{article}
\setlength{\oddsidemargin}{0in}
\setlength{\evensidemargin}{0in}
\setlength{\textwidth}{6.5in}
\setlength{\parindent}{0in}
\setlength{\parskip}{\baselineskip}
\usepackage{amsmath,amsfonts,amssymb}
\usepackage{graphicx}
\usepackage[]{algorithmicx}
\usepackage{enumitem}
\usepackage{fancyvrb}
\usepackage{setspace}

\usepackage{fancyhdr}
\pagestyle{fancy}
\setlength{\headsep}{36pt}

\usepackage{hyperref}

\hypersetup{
    colorlinks=true,
    linkcolor=blue,
    filecolor=magenta,      
    urlcolor=blue,
}

\newcommand{\makenonemptybox}[2]{%
%\par\nobreak\vspace{\ht\strutbox}\noindent
\item[]
\fbox{% added -2\fboxrule to specified width to avoid overfull hboxes
% and removed the -2\fboxsep from height specification (image not updated)
% because in MWE 2cm is should be height of contents excluding sep and frame
\parbox[c][#1][t]{\dimexpr\linewidth-2\fboxsep-2\fboxrule}{
  \hrule width \hsize height 0pt
  #2
 }%
}%
\par\vspace{\ht\strutbox}
}
\makeatother

\begin{document}
\lhead{{\bf CSCI 3104, Algorithms \\ Final Exam Summer 2020 (60 points)  } }
\rhead{Name: \fbox{% Place your name here and delete the next time
\phantom{This is a really long name}} 
\\ ID: \fbox{ %+ Place your ID here and delete the next time
\phantom{This is a student ID}} 
\\ {\bf Escobedo \& Jahagirdar\\ Summer 2020, CU-Boulder}}
\renewcommand{\headrulewidth}{0.5pt}

\phantom{Test}

\begin{figure}[h!]
    \begin{center}
    \includegraphics[scale=0.15]{Final/dont_panic.png} 
    \end{center}
    \end{figure}
\begin{small}


\textit{Advice 1}:\ For every problem in this class, you must justify your answer:\ show how you arrived at it and why it is correct. If there are assumptions you need to make along the way, state those clearly.
%\vspace{-3mm} 

\textit{Advice 2}:\ Verbal reasoning is typically insufficient for full credit. Instead, write a logical argument, in the style of a mathematical proof.\\
%\vspace{-3mm} 

\textbf{Honor code}: On my honor as a University of Colorado at Boulder student,
I have neither given nor sought unauthorized assistance in this work\\  \\
\textbf{Initials} 
\fbox{% Place your iniitals here and delete the next time
\phantom{This is a really long name}} \\ \\
\textbf{Date} 
\fbox{% Place your Date here and delete the next time
\phantom{This is a really long name}} \\ 


If you violate the CU \textbf{Honor Code}, you will receive a 0.
\clearpage
\textbf{Instructions for submitting your solution}:
\vspace{-5mm} 

\begin{itemize}
	\item The solutions \textbf{should be typed}, we cannot accept hand-written solutions. Here's a short intro to \href{http://ece.uprm.edu/~caceros/latex/introduction.pdf}{\textbf{Latex}.}
	 \item In this homework we denote the asymptomatic \textit{Big-O} notation by $\mathcal{O}$ and \textit{Small-O} notation is represented as $o$. 
	\item We recommend using online Latex editor \href{https://www.overleaf.com/}{\textbf{Overleaf}}. Download the \textbf{.tex} file from Canvas and upload it on overleaf to edit.
	%todo add link of gradescope
	\item You should submit your work through \href{https://www.gradescope.com}{\textbf{Gradescope}}  only.
	\item If you don't have an account on it, sign up for one using your CU email. You should have gotten an email to sign up. If your name based CU email doesn't work, try the identikey@colorado.edu version. 
	\item Gradescope will only accept \textbf{.pdf} files (except for code files that should be submitted separately on Canvas if a problem set has them) and \textbf{try to fit your work in the box provided}. 
	\item You cannot submit a pdf which has less pages than what we provided you as Gradescope won't allow it.
   
\end{itemize}
\vspace{-4mm} 
\end{small}
\clearpage

%\hrulefill





\begin{enumerate}


	
	
	
	\item{
	    (5 pts) You are given a list of jobs with start and end times. Assume that only one job can be executed at a time on a processor. Your task is to find out the minimum number of processors required so that the jobs are  executed as soon as they arrive (no waiting time for a process).
	    The input to your algorithm will be a list of tuples indicating the start and end times of the respective jobs. The output should be a number indicating the minimum number of processors required. Your algorithm should have a runtime of $\mathcal{O}(n \log(n))$, where $n$ is the number of jobs. Assume you do not have access to any auxiliary functions.\\
	    
	    \textbf{Example} \\
        \textbf{Input}: $(2, 10) , (9, 11), (15, 18), (3, 4), (17, 19), (5, 13)$ \\
        \textbf{Output}: $3$ \\
        \textbf{Explanation}: Between the interval $(9, 10)$ it can be seen that there will be 3 \{ (2, 10) , (9, 11),  (5, 13)\} jobs that need to be executed simultaneously. Thus at the very least 3 processors are required. \\ 
        
        \begin{enumerate}
	        \item (4 pts) Write down well commented pseudo-code or paste real code to solve the above problem.
	        \makenonemptybox{3in}{}
	        \clearpage
	        \item (1 pts) Explain your algorithms runtime and space complexity by analyzing your code. For example, stating that a sorting algorithm runs in $\mathcal{O}(n \log(n))$ without any justification is insufficient. 
	        \makenonemptybox{3in}{}
	    \end{enumerate}
	}
	
	\clearpage
	
	\item{
	    (5 pts) You are designing a network, to connect several offices in different locations. You can contact your network cable provider about the cost of connecting any pair of offices. Your task is to design an algorithm, so that all offices are connected and the cost associated with building the network is minimized.\\
	    The input to your algorithm is a Graph in the form of an adjacency list or matrix. The nodes in this graph represent the offices and edge weights represent the cost associated with connecting a pair of offices. The output of the algorithm will be a list of office pairs such that the total connection cost is minimized. \\
	    \begin{enumerate}
	        \item (2 pts) Provide an example with at least 3 offices and 3 connections. Your example must include at least 2 unique solutions.
	        \makenonemptybox{4in}{}
	        \clearpage
	        
	        \item (3 pts) Provide well commented pseudo code or actual code to solve the above problem.
	        \makenonemptybox{6in}{}
	    \end{enumerate}
	   
	}
	
	\clearpage
	
	\item{
	    (10 pts) Consider the graph $G = (V, E)$ with edge capacities $c(e), \; \forall e \in E$, the edge capacity represents the maximum amount of flow that can pass through the edge. In addition to edge capacities, the above graph has vertex capacities $c(v), \; \forall v \in V - \{s, t\}$ (The source and the sink vertex do not have vertex capacities). Each vertex capacity represents the
	    maximum amount of flow that can pass through the vertex. \\
	     
	    
	    \begin{enumerate}
	        \item (4 pts) Given a source vertex $s \in V$ and a sink $t \in V$.  How will you modify the given graph $G$, so that you can find the max-flow from $s$ to $t$ with a known algorithm? Also, show an example input and modified graph with at least 5 vertices. This must be a drawing! 
	        \makenonemptybox{4in}{}
	        \clearpage
	        
	        \item (4 pts) Provide an algorithm to find the max-flow in the above graph from $s$ to $t$.
	        \makenonemptybox{6in}{}
	        \clearpage
	        
	        \item (2 pts) Find the max-flow in your example graph by stepping though your algorithm. Provide a new graph image each time the flow though an edge is altered. 
	        \makenonemptybox{3in}{}
	        
	    \end{enumerate}
	}
	
	\clearpage
	\item{
	    
	    (10 pts) There are $n$ rooms numbered from $\{1, 2, .. n\}$ in  Williams Village. Each of the rooms has a teleportation device with a value $v_i \; \; \forall i \in \{1, 2, ....n\}$. The teleportation device placed in the room $i$ with value $v_i$ teleports to a room numbered  $(v_i \; \% \; n) + 1$. 
	    
	    A room in Williams Village is said to be beautiful if you can get back to the room you started using the teleportation devices. The task is to count the number of beautiful rooms in Williams Village. \\
	    The input to your algorithm is the list of values associated with the teleportation device placed in the rooms. The output should be a number indicating the number of beautiful rooms. \\ \\
	    
	    \textbf{Example 1}\\
	    \textbf{Input}: [2, 3, 4, 5] \\
	    \textbf{Output}: 4\\
	    \textbf{Explanation}: All the 4 rooms that are beautiful and the corresponding paths  to get back to them are as follows.\\
	    \begin{itemize}
	        \item \textbf{Room 1} The teleportation path for Room1 is Room1, Room3, and back to Room1
	        \item \textbf{Room 2} The teleportation path for Room2 is Room2, Room4, and back to Room2
	        \item \textbf{Room 3} The teleportation path for Room3 is Room3, Room1, and back to Room3
	        \item \textbf{Room 4} The teleportation path for Room4 is Room4, Room2, and back to Room4
	    \end{itemize}
	    
	    \textbf{Example 2}\\
	    \textbf{Input}: [1, 2, 3, 6] \\
	    \textbf{Output}: 2\\
	    \textbf{Explanation}: It can be seen that only Room3 and Room4 are beautiful.
	    
	    \clearpage
	    \begin{enumerate}
	        \item (2 pts) Give a high-level explanation of how you plan to solve this problem. (How would you explain your potential solution to a younger sibling or someone who has never taken a computer science class?) Minimum 1 paragraph.
	        \makenonemptybox{3in}{}
	        
	        \clearpage
	        \item (5 pts) Provide well commented pseudo code or actual code to solve the above problem.
	        \makenonemptybox{6in}{}
	        
	        \clearpage
	        \item (3 pts) Give an example input with at least 5 rooms and structure of how your algorithm explores the rooms. This must be an image! Your input can \textbf{not} be an example already given.
	        \makenonemptybox{3in}{}
	    \end{enumerate}
	}
	
	\clearpage
	
	\item{
	    (10 pts) You are at your home in Boulder enjoying the summer and your friend challenges you to ride your bike for at least $s$ miles starting from your home and ending at any of the scenic outlooks. You have a map of Boulder that has information about roads connecting various view points and your house. The task is to determine if there is a sequence of scenic outlooks that you can travel to starting from your house such that you have travelled at least $s$ miles. You can not travel on the same edge more than 1 time and once you visit an outlook you cannot travel to it again.\\
	    The input to your problem is a graph $G=(V, E)$  in the form of an adjacency list or adjacency matrix and the source vertex $s$ corresponding to your house. Your algorithm should output a boolean value indicating if it's possible to complete the challenge of riding at least $s$ miles starting from your house.
	    
	    \begin{enumerate}
	        \item (2 pts) Give a high-level explanation of how you plan to solve this problem. (How would you explain your potential solution to a younger sibling or someone who has never taken a computer science class?) Minimum 1 paragraph. 
	        \makenonemptybox{4in}{}
	        \clearpage
	        
	        \item (6 pts) Provide well commented pseudo code or actual code to solve the above problem.
	        \makenonemptybox{6in}{}
	        \clearpage
	        
	        \item (2 pts) Explain your algorithms runtime and space complexity by analyzing your code. For example, stating that a sorting algorithm runs in $\mathcal{O}(n \log(n))$ without any justification is insufficient. 
	        \makenonemptybox{3in}{}
	        \clearpage
	    \end{enumerate}
	}
	
	\clearpage 
		\item{
    (20 pts) There are an even number of items arranged along a line. Each item has a value $v_i$ associated with it. You are competing against an opponent to collect items such that total value of your items collected is maximized. At each round of the game, one player is allowed to pick an item from either end of the line. \\ \\
    Determine the maximum total value that can be obtained, if you are allowed to play first assuming that the opponent is equally clever. The input to your algorithm will be a list of values. The output should be a number indicating the maximum amount of value that you can obtain.\\ 
    
    \textbf{Example} \\
    \textbf{Input}: $22, 43, 10, 20$ \\
    \textbf{Output}: $63$ \\
    \textbf{Explanation}: As the first player I would first pick 20 from the right end. The opponent would then pick 22. Out of the two elements 43, 10 remaining, I can pick 43. Thus 20+43=63 is the maximum total value that can be obtained. \\
    \textbf{Note}: Observe in the above example that the greedy choice of picking the max valued item does not work. If I had picked 22 instead of 
    20 in the first round, the maximum value that I could have obtained is 32.
    
    \begin{enumerate}
        \item (5 pts) State the base case and recursive relation that can be used to solve the above problem using dynamic programming.
        \makenonemptybox{3in}{}
        \item (5 pts) Provide an example input consisting of at least 5 items. Show how the dynamic programming data structure used to store previous computations is built based on the recursive relation. 
        \makenonemptybox{5in}{}        
        \clearpage

        \item (8 pts) Write down well commented pseudo-code or paste real code to solve the above problem using Dynamic Programming or Memoization based on the above recurrence relation.
        \makenonemptybox{6in}{}
        \clearpage
        
        
        \item (2 pts) Discuss the space and runtime complexity of the code, providing necessary justification.
        \makenonemptybox{3in}{}
    \end{enumerate}  
	} 
	\clearpage
	
	\pagebreak
    
    
    \item{ \textbf{Extra Credit (3 pts) will only be considered if your final exam score is less than 100\% \\}
    \begin{doublespace}
    Write two double spaced pages about the job or position you hope to have when you graduate from college (If you already have a job lined up, write about your next career goal). Your essay must include i) a tractable career goal, ii) a reason why you want to be hired/accepted to this position, iii) a step-by-step plan of how you intend to achieve your goal. 
    \end{doublespace}
    }
    
    

    
	
\end{enumerate}


\end{document}


